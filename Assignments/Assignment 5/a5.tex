\documentclass[12pt]{article}

\usepackage{color}
\usepackage{graphicx}
\usepackage{amsmath}
\usepackage{float}
\usepackage{color}
\usepackage{indentfirst}
\usepackage{textcomp}
\usepackage{pifont}
\usepackage{multirow}
\usepackage{geometry}
\usepackage{algorithm}
\usepackage{algpseudocode}
\usepackage{amssymb}
\usepackage{algorithmicx}
\usepackage{amsmath} 
\usepackage{amsfonts}
\usepackage{bm}
\usepackage{enumitem}
\usepackage{multirow}

\usepackage{listings}
\usepackage{xcolor}

\geometry{left = 2cm, right = 2cm, top = 2.5cm, bottom = 2.5cm}

\lstset{numbers=left,
        numberstyle=\tiny,
        keywordstyle=\color{blue}, 
        commentstyle=\color[cmyk]{1,0,1,0}, 
        frame=single, 
        escapeinside=``, 
        extendedchars=false, 
        xleftmargin=2em,xrightmargin=2em, aboveskip=1em, 
        tabsize=4, 
        showspaces=false
       }

\floatname{algorithm}{Algorithm}
\renewcommand{\algorithmicrequire}{\textbf{Input:}}
\renewcommand{\algorithmicensure}{\textbf{Output:}}





\begin{document}

\vspace*{0.25cm}

\hrulefill

\thispagestyle{empty}

\begin{center}
\begin{large}
\sc{UM--SJTU Joint Institute \vspace{0.3em} \\ Bayesian Analysis \\ (VE414)}
\end{large}

\hrulefill

\vspace*{5cm}
\begin{Large}
\sc{{Assignment 5 \\ ~\\ }}
\end{Large}
\vspace{2em}

\end{center}


\vfill
\begin{large}

\begin{table}[h!]
\flushleft
\begin{tabular}{lll}
Name: Wu Guangzheng \hspace*{2em}\\
Student ID: 515370910014

\end{tabular}
\end{table}
\end{large}
\newpage
\begin{flushleft} 

\section{Question 1}

\subsection*{a)}

\qquad With the results in the last assignment, the constant $c$ is $0.645$, that is 

$$
f_{Y\mid X} = 0.645 \times \exp(-\frac{(x-y)^2}{2}) \times \frac{1}{1+y^2}
$$

\qquad And we have

$$
0 < \frac{1}{1+y^2} \leq 1, \qquad y \in R
$$

\qquad So 

$$
f_{Y\mid X} \leq 0.645 \times \exp(-\frac{(x-y)^2}{2}) = 0.645 \cdot \sqrt{2\pi \sigma^2} \cdot \frac{1}{\sqrt{2\pi \sigma^2}} \cdot \exp(-\frac{(y-\mu)^2}{2\sigma^2})
$$

\qquad Where

$$
\sigma = 1, \qquad \mu = x = 0.5
$$

\qquad So we have

$$
f_{Y\mid X} \leq AMq(y) < Mq(y)
$$
\vspace{-0.5cm}
$$
A = 0.645,  \qquad M = \sqrt{2\pi}, \qquad q(y) \sim \text{Norm}(0.5, 1)
$$

\subsection*{b)}

\qquad We can get the following result

\begin{table}[h]
\centering
\begin{tabular}{cc}
\hline
Grid Size $n$ & $\mathbb{E}\left[Y\mid X = 0.5\right]$ \\
\hline
50    &  0.2059457941459266  \\
250   &  0.26953845572240126 \\
750   &  0.2728096525929137  \\
1500  &  0.2800835653746651  \\
3000  &  0.2585988667944461  \\
\hline
\end{tabular}
\end{table}

\qquad With code

\begin{lstlisting}[language=R]
using Distributions

sample_sizes = [50, 250, 750, 1500, 3000]
    
for n in sample_sizes 
    samples = Array{Float64}(undef, n)
    i = 1
    while i <= n
        v = rand(1)[1]
        y = rand(Normal(0.5, 1), 1)[1]
        if v <= 1 / (1+y^2)
            samples[i] = y
            i+=1 
        end
    end
    exp = sum(samples) / n
    println(exp)
end        
\end{lstlisting} 

\subsection*{c)}

\qquad We can use the same $g(y)$ here, which is

$$
q(y) \sim \text{Normal}(0.5, 1)
$$

\qquad So we have

\begin{align*}
E\left[Y\mid X\right] &= \int Y \cdot f_{Y\mid X} dy = \int \frac{Y \cdot f_{Y\mid X}}{q(y)} \cdot q(y) dy\\
&\approx \frac{1}{n} \sum_{i=1}^{n} \frac{p(y^{(i)})}{q(y^{(i)})}\cdot y^{(i)}\\
&= \frac{1}{n} \sum_{i=1}^{n} \frac{f_{Y\mid X}(y^{(i)})}{q(y^{(i)})}\cdot y^{(i)}\\
&= \frac{1}{n} \sum_{i=1}^n \sqrt{2\pi} \cdot \frac{y}{1+y^2}
\end{align*}

\subsection*{d)}

\qquad We can get the following result

\begin{table}[h]
\centering
\begin{tabular}{cc}
\hline
Grid Size $n$ & $\mathbb{E}\left[Y\mid X = 0.5\right]$ \\
\hline
50    &  0.31778840599136227  \\
250   &  0.2931960705649614 \\
750   &  0.2853944180824812  \\
1500  &  0.2609568019578496  \\
3000  &  0.2758034794541073 \\
\hline
\end{tabular}
\end{table}

\newpage

\qquad With code

\begin{lstlisting}[language=R]
using Distributions

sample_sizes = [50, 250, 750, 1500, 3000]
    
for n in sample_sizes 
    samples = Array{Float64}(undef, n)
    wi = Array{Float64}(undef, n)
    i = 1
    while i <= n
        v = rand(1)[1]
        y = rand(Normal(0.5, 1), 1)[1]
        w = sqrt(2 * pi) * y / (1 + y^2)
        wi[i] = sqrt(2 * pi) / (1 + y^2)
        samples[i] = w
        i+=1 
    end
    exp = sum(samples) / sum(wi)
    println(exp)
end    
\end{lstlisting} 

\section{Question 2}

\qquad According to Box Muller Transform, if $x_1$ and $x_2$ are uniform random samples on the interval $(0,1)$, and

$$
z_1 = \sqrt{-2 \ln x_1}\cos(2\pi x_2), \qquad z_2 = \sqrt{-2 ln x_1}sin(2\pi x_2)
$$

\qquad then $z_1$ and $z_2$ are random samples follows $\text{Normal}(0,1)$.

~\\

\qquad Then we can sample any Normal distribution by the following equation, then we can sample any Normal Distribution.

$$
\text{Normal}(\mu, \sigma^2) = \mu + \sigma \cdot \text{Normal}(0,1)
$$

\qquad As we have

$$
f_{Y_1Y_2}(y_1, y_2) = \frac{1}{2\pi \sqrt{1-\rho^2}}\exp\left(-\frac{y_1^2 - 2\rho y_1y_2 + y_2^2}{2(1-\rho^2)}\right)
$$

\qquad And

$$
f_{Y_1}(y_1) = \frac{1}{\sqrt{2\pi}}\exp(-\frac{y_1^2}{2}), \qquad f_{Y_2}(y_2) = \frac{1}{\sqrt{2\pi}}\exp(-\frac{y_2^2}{2})
$$

\qquad So

$$
f_{Y_1 \mid Y_2 = y_2}(y_1) = \frac{f_{Y_1Y_2}(y_1, y_2)}{f_{Y_2}(y_2)} = \frac{1}{\sqrt{2\pi (1-\rho^2)}}\exp(-\frac{(y_1 - \rho y_2)^2}{2(1-\rho^2)}) \sim \text{Normal}(\rho y_2, 1-\rho^2)
$$

\qquad And similarly

$$
f_{Y_2 \mid Y_1 = y_1}(y_2) \sim \text{Normal}(\rho y_1, 1-\rho^2)
$$

\qquad Based on all the things above, we can use a Gibbs Sampling that

\vspace{-0.5cm}

\begin{align*}
&P(Y_1^{t} \mid Y_2^{t-1} = y_2) \sim \text{Normal}(\rho y_2, 1-\rho^2)\\
&P(Y_2^{t} \mid Y_1^{t} = y_1) \sim \text{Normal}(\rho y_1, 1-\rho^2)
\end{align*}

\qquad Where $Y_1^{0}$ and $Y_2^{0}$ are both sampled from a unifrom distribution.

\section{Question 3}

\qquad When the proposal distribution is the corresponding conditional distribution, 


\newpage

























\subsection*{a)}

\qquad Let 

$$
u = g(y) = \arctan y \qquad v = h(y) = \arctan y
$$

\qquad So 

$$
y = g^{-1}(u) = \tan u \qquad y = h^{-1}(v) = \tan v
$$

\qquad So we have

\vspace{-0.5cm}

\begin{align*}
\int_{-\infty}^{\infty}f_{Y\mid X} dy &= \int_{-\infty}^{0} f_{Y\mid X}dy + \int_{0}^{\infty} f_{Y\mid X}dy\\
&= \int_{-\infty}^{0} \exp(-\frac{(x-y)^2}{2})\cdot \frac{1}{1+y^2}dy + \int_{0}^{\infty} \exp(-\frac{(x-y)^2}{2})\cdot \frac{1}{1+y^2}dy\\
&= \int_{-\frac{\pi}{2}}^0 \exp(-\frac{(x-\tan u)^2}{2}) du + \int_{0}^{\frac{\pi}{2}} \exp(-\frac{(x-\tan v)^2}{2}) dv\\
&\approx \frac{\pi}{2n}\sum_{i=1}^n \exp(-\frac{(x - \tan u_i)^2}{2}) + \frac{\pi}{2n}\sum_{i=1}^n \exp(-\frac{(x - \tan v_i)^2}{2})\\
&= \frac{\pi}{2n} \sum_{i=1}^{n} \exp(-\frac{(x-\tan(-\frac{\pi}{2} + \frac{\pi (i-1)}{2n}))^2 }{2}) + \frac{\pi}{2n} \sum_{i=1}^{n} \exp(-\frac{(x-\tan(\frac{\pi (i-1)}{2n}))^2 }{2})
\end{align*}

\qquad Here we define the following, in order to ensure the continuity

$$
\tan -\frac{\pi}{2} = -\infty \qquad \text{and} \qquad  \tan \frac{\pi}{2} = \infty
$$

\subsection*{b)}

\qquad So the expectation should be

\begin{align*}
\mathbb{E} \left[ Y\mid X \right] &= \int_{-\infty}^{\infty} y \cdot f_{Y\mid X} dy\\
& \approx \frac{\pi}{2n}\sum_{i=1}^n \tan u_i \cdot \exp(-\frac{(x - \tan u_i)^2}{2}) + \frac{\pi}{2n}\sum_{i=1}^n \tan v_i \cdot \exp(-\frac{(x - \tan v_i)^2}{2})
\end{align*}

\qquad Where $u_i$ and $v_i$ are the linspace values dividing $(-\frac{\pi}{2}, 0)$ and $(0, \frac{\pi}{2})$ into $n-1$ parts.

\newpage

\qquad So we can calculate the value by Julia, with the following code,

\begin{lstlisting}[language=R]

n = [50, 250, 750, 1500, 3000]
for ni in n
    ni = round(Int, ni/2)
    u = range(-1 * pi / 2, length = ni+1, stop = 0)
    v = range(0, length = ni+1, stop = +1 * pi / 2)
        
    u = collect(u[2:ni+1])
    v = collect(v[1: ni])
        
    u_tan = map((x)->tan(x), u)
    v_tan = map((x)->tan(x), v)
        
    u_exp = map((x) -> exp(-((0.5 - x)^2)/2), u_tan)
    v_exp = map((x) -> exp(-((0.5 - x)^2)/2), v_tan)
        
    inte = pi / (2*ni) * sum(u_exp) + pi/(2*ni) * sum(v_exp)
    c = 1 / inte
    E = c * pi / (2*ni) * sum(u_tan .* u_exp) + 
        c * pi/(2*ni) * sum(v_tan .* v_exp)
    println(E)
end        
\end{lstlisting} 
    

\qquad And we get the result

\begin{table}[h]
\centering
\begin{tabular}{cc}
\hline
Grid Size $n$ & $\mathbb{E}\left[Y\mid X = 0.5\right]$ \\
\hline
50    &  0.2569800409221861  \\
250   &  0.264284665679843   \\
750   &  0.2655426364320458  \\
1500  &  0.2658590025448422  \\
3000  &  0.2660174684574401  \\
\hline
\end{tabular}
\end{table}

\subsection*{c)}

\qquad We use the former grid approximation, and apply a direct sampling scheme on it.

\begin{lstlisting}[language=R]
using Distributions
n = [50, 250, 750, 1500, 3000]
        
for ni in n
    ni = round(Int, ni/2)
    u = range(-1 * pi / 2, length = ni+1, stop = 0)
    v = range(0, length = ni+1, stop = +1 * pi / 2)
    weights = Array{Float64, 1}(undef, ni)
    for x in 1:ni
        weights[x] = 1/ni
    end
            
    u = collect(u[2:ni+1])
    v = collect(v[1: ni])
        
    m = [100, 1000]
        
    for mi in m
        um = wsample(u, weights, mi)
        vm = wsample(v, weights, mi)
            
        u_tan = map((x)->tan(x), um)
        v_tan = map((x)->tan(x), vm)
        
        u_exp = map((x) -> exp(-((0.5 - x)^2)/2), u_tan)
        v_exp = map((x) -> exp(-((0.5 - x)^2)/2), v_tan)
        
        inte = pi / (2*ni) * sum(u_exp) + pi/(2*ni) * sum(v_exp)
        c = 1 / inte
        E = c * pi / (2*ni) * sum(u_tan .* u_exp) 
                + c * pi/(2*ni) * sum(v_tan .* v_exp)
        println(ni*2, " ", mi, " ", E)
    end
end
\end{lstlisting}

\subsection*{d)}

\qquad From the code above, we get the following results

\begin{table}[h]
\centering
\begin{tabular}{ccc}
        \hline
        Grid Size $n$ & m & $\mathbb{E}\left[Y\mid X = 0.5\right]$ \\
        \hline
        \multirow{2}*{50}    & 100  & 0.26858755239311605    \\
        ~                    & 1000 & 0.2430165559582972     \\
        \multirow{2}*{250}   & 100  & 0.26221830883887703    \\
        ~                    & 1000 & 0.2574684284198875     \\
        \multirow{2}*{750}   & 100  & 0.24911899520477043    \\
        ~                    & 1000 & 0.2687289119332303     \\
        \multirow{2}*{1500}  & 100  & 0.28066283942057557    \\
        ~                    & 1000 & 0.2622935268454715     \\
        \multirow{2}*{3000}  & 100  & 0.26459162967251004    \\
        ~                    & 1000 & 0.2716451740382342     \\
        \hline
\end{tabular}
\end{table}

\subsection*{e)}

\qquad We modify the code so that we can use the code directly to calculate the expectation:

\begin{lstlisting}[language=R]
ni = [50, 250, 750, 1500, 3000]

for n in ni
    x = 0.5 
    a = -5
    b = 5
    if n <= 1000
        y_grid = collect(range(a, length=n, stop=b));
        newa = a;
    elseif n > 1000 && n <= 2000
        nm = 1000;
        na=round(Int, (n-nm)/2);
        l = (b-a)/(nm-1); 
        newa = a-l*na;
        y_grid = collect(range(newa, step=l , length=n));
    else n > 2000 
        nm=round(Int, n/2);
        na=round(Int, (n-nm)/2); l = (b-a)/(nm-1);
        newa = a-l*na;
        y_grid = collect(range(newa, step=l , length=n));
    end
    dy = y_grid[3] - y_grid[2]
    unnormalized_posterior = Array{Float64, 1}(undef, n)
    for i in 1:n
        unnormalized_posterior[i] = y_grid[i] 
            * exp(-(x-y_grid[i])^2/2) * (1/(1+y_grid[i]^2)) * dy
    end
    C = 1/1.549
    println(sum(unnormalized_posterior) * C)
end   
\end{lstlisting} 

\qquad And we get the result

\begin{table}[h]
\centering
\begin{tabular}{cc}
\hline
Grid Size $n$ & $\mathbb{E}\left[Y\mid X = 0.5\right]$ \\
\hline
50    &  0.2661755641372646    \\
250   &  0.26617525294322397   \\
750   &  0.26617519291542174   \\
1500  &  0.26617612339045527   \\
3000  &  0.26617612339069857   \\
\hline
\end{tabular}
\end{table}

\end{flushleft}
\end{document}